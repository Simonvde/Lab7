%%%%%%%%%%%%%%%%%%%%%%%%%%%%%%%%%%%%%%%%%
% Short Sectioned Assignment
% LaTeX Template
% Version 1.0 (5/5/12)
%
% This template has been downloaded from:
% http://www.LaTeXTemplates.com
%
% Original author:
% Frits Wenneker (http://www.howtotex.com)
%
% License:
% CC BY-NC-SA 3.0 (http://creativecommons.org/licenses/by-nc-sa/3.0/)
%
%%%%%%%%%%%%%%%%%%%%%%%%%%%%%%%%%%%%%%%%%

%----------------------------------------------------------------------------------------
%	PACKAGES AND OTHER DOCUMENT CONFIGURATIONS
%----------------------------------------------------------------------------------------

\documentclass[paper=a4, fontsize=11pt]{scrartcl} % A4 paper and 11pt font size

\usepackage[T1]{fontenc} % Use 8-bit encoding that has 256 glyphs
%\usepackage{fourier} % Use the Adobe Utopia font for the document - comment this line to return to the LaTeX default
\usepackage[english]{babel} % English language/hyphenation
\usepackage[utf8]{inputenc}  %allows non-English characters
\usepackage{amsmath,amsfonts,amsthm} % Math packages
\usepackage{float}

\usepackage{sectsty} % Allows customizing section commands
%\allsectionsfont{\centering \normalfont\scshape} % Make all sections centered, the default font and small caps
\allsectionsfont{\centering}

\usepackage{fancyhdr} % Custom headers and footers
\pagestyle{fancyplain} % Makes all pages in the document conform to the custom headers and footers
\fancyhead{} % No page header - if you want one, create it in the same way as the footers below
\fancyfoot[L]{} % Empty left footer
\fancyfoot[C]{} % Empty center footer
\fancyfoot[R]{\thepage} % Page numbering for right footer
\renewcommand{\headrulewidth}{0pt} % Remove header underlines
\renewcommand{\footrulewidth}{0pt} % Remove footer underlines
\setlength{\headheight}{13.6pt} % Customize the height of the header

%\usepackage{geometry}
%\usepackage{pdflscape}


%\numberwithin{equation}{section} % Number equations within sections (i.e. 1.1, 1.2, 2.1, 2.2 instead of 1, 2, 3, 4)
%\numberwithin{figure}{section} % Number figures within sections (i.e. 1.1, 1.2, 2.1, 2.2 instead of 1, 2, 3, 4)
%\numberwithin{table}{section} % Number tables within sections (i.e. 1.1, 1.2, 2.1, 2.2 instead of 1, 2, 3, 4)

%\setlength\parindent{0pt} % Removes all indentation from paragraphs - comment this line for an assignment with lots of text

\usepackage{caption}
%\usepackage{topcapt}

\usepackage{booktabs}

\usepackage{graphicx}
\usepackage{adjustbox}
\graphicspath{{../images/}}


%shortcuts for typing variance and expectation
\newcommand{\E}{\mathrm{E}}
\newcommand{\Var}{\mathrm{Var}}

%----------------------------------------------------------------------------------------
%	TITLE SECTION
%----------------------------------------------------------------------------------------

\newcommand{\horrule}[1]{\rule{\linewidth}{#1}} % Create horizontal rule command with 1 argument of height

\title{
\normalfont \normalsize
\textsc{UPC - Complex and Social Networks} \\ [25pt] % Your university, school and/or department name(s)
\horrule{0.5pt} \\[0.4cm] % Thin top horizontal rule
\huge Lab7: Simulation of SIS model over networks \\ % The assignment title
\horrule{2pt} \\[0.5cm] % Thick bottom horizontal rule
}

\author{Mart\'{i} Renedo\and Simon Van den Eynde} % Your name

\date{\normalsize\today} % Today's date or a custom date

\begin{document}


\maketitle % Print the title


%----------------------------------------------------------------------------------------
%	INTRO
%----------------------------------------------------------------------------------------

\section{Introduction}

We will simulate a SIS epidemic model on networks. We will use a binary tree, a scalefree graph, an Erdos-Renyi graph, a complete graph and a star graph. We will also look at their thresholds (the inverse of the highest eigenvalue of the graph) and try to see how they influence the epidemic behaviour.


\section{Results}

The following graphs are the result of a simulation on $5$ different graphs with $100$ timesteps. For every graph we took $\gamma=.1$, then we calculated the thresholdvalue for $\beta$. We increased $\beta$ by $10\%$ (plotted dots) and decreased $\beta$ by $10\%$ (plotted red line). We started with $5\%$ infected

\begin{figure}[htbp] %  figure placement: here, top, bottom, or page
   \centering
   \includegraphics[width=\textwidth]{thresholdSimulation_tree} 
   \label{tree}
\end{figure}
\begin{figure}[htbp] %  figure placement: here, top, bottom, or page
   \centering
   \includegraphics[width=\textwidth]{thresholdSimulation_scaleFree} 
   \label{scaleFree}
\end{figure}
\begin{figure}[htbp] %  figure placement: here, top, bottom, or page
   \centering
   \includegraphics[width=\textwidth]{thresholdSimulation_erdosRenyi} 
   \label{erdosRenyi}
\end{figure}
\begin{figure}[htbp] %  figure placement: here, top, bottom, or page
   \centering
   \includegraphics[width=\textwidth]{thresholdSimulation_complete} 
   \label{complete}
\end{figure}
\begin{figure}[htbp] %  figure placement: here, top, bottom, or page
   \centering
   \includegraphics[width=\textwidth]{thresholdSimulation_star} 
   \label{star}
\end{figure}

\section{Discussion}
We notice that we see we crossed a threshold value for the complete graph, the Erdos-Renyi graph and the scalefree graph. The star graph, tree, however, do not seem to have crossed any threshold. If we would 

From this, we can conclude that the theory doesn't hold for some graph, in particular it doesn't hold for the trees we tested it on. To be sure it does hold for the complete, Erdos-Renyi and scalefree graph, we would have to do some more tests (values closer to the threshold, different numbers of vertices, run the simulation longer), but we are confident there is a threshold here.

\section{Methods}
To find the thresholds we used ARPACK, a package partially built in in igraph. It has fast methods for calculating the largest eigenvalue of a graph. Calculating the $5$ eigenvalues of the $5$ graphs with $2000$ vertices only takes seconds.

For generating the graphs and plots we set a seed so it can be easily replicated.

We chose the parameters for the Erdos-Renyi and scalefree graph so that they both had about $20000$ edges.

\end{document}